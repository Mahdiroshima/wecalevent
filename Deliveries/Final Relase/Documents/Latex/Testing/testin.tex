\section{Testing}
\subsection{General approach}
\par In order to test our project, we decided to create a manual test for each functionality and to use some automated tests. The performed tests will check the match between our application and the specifications written in requirements document.
\subsection{Environment}
we use test cases to be sure about our application stability. To perform this task we decide to use Junit.  Its a simple Java tool that allow us to write an understandable and easy manner the unit tests.
\subsection{Features to be tested}
\par \begin{itemize}
       \item Registration
       \item Login/Logout
       \item Create an event
       \item Update an event
       \item Delete an event
       \item Answer event invitation
       \item Search users
       \item Update profile
       \item Import/Export calendar
       \item Notify users
     \end{itemize}
\subsection{Features not to be tested}
\par Search an event feature will not be tested as we decided not to have it in our application, as all events can be found while viewing calendars. 
\subsection{Test Case Report}
 \begin{center} \begin{tabular}{|l|l|}
  \hline
  Test Case ID& The ID or the number of Test Case\\
  \hline
  Test Case Description & Description of the test case\\
  \hline
  Actors Involved & The actors of this Test Case\\
   \hline
  Precondition & The state of the system at the time the test starts\\
  \hline
  Main path &  List of steps that need to be applied for this Test Case \\
  \hline
  Expected result & The result that should be observed from a successful test.\\
  \hline
  Conclusion & \begin{minipage}{5in}
    \vskip 4pt
               \begin{itemize}
                 \item Failed
                 \item Successful
               \end{itemize}

     \vskip 4pt
  \end{minipage}
  \\
  \hline
\end{tabular} \end{center}

