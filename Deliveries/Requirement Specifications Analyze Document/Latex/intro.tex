\newpage
\begin{center}
\section{Introduction}
\end{center}

\subsection{Description}
\quad WeCalEvent is a platform where users are able to create and manage their events and invite other users. Information about time and place of the event should be provided while creation of the event. The system alerts the event creator 3 days before the event day in case of weather condition alteration and suggests him the closest day with suitable weather.
\par Invited users will be able to decline or accept the invitation. All event participants will be notified in case of weather forecast alteration a day before the event.
\par This platform is sufficient for users who look forward to schedule their events according to the weather conditions.

\subsection{Purpose}
\quad The purpose of the Requirements Analysis and Specification Document is to clearly state every requirement, functions and features of the product. It is important to predict and sort out how we hope this product will be used. RASD will provide a detailed overview of our software product, its parameters and goals, being useful both for developers and customers. Therefore, this document is written for mainly developers of the system to make sure that all team members has the same vision of requirements. Secondarily, it is written for the Software Engineering 2 course staff to give them general description of our project requirements.

\subsection{Scope}
\quad The project named WeCalEvent which our team intend to develop is a web application that aims to provide Event management environment in the basis of weather forecast. The users of our system can easily create events invent other users who currently use our system. Basically, our application is capable of dealing with bad conditions. For example, the system notifies and suggest alternative dates with good weather conditions to the user in case of bad weather conditions on the current date of the event.
\subsection{Definitions, acronyms, and abbreviations}
\quad \par \textbf{Weather forecast:} a statement saying what the weather will be like the next day, or the next few days.
\par \textbf{Calendar:} is a table or register with the days of each month and week in a year.
\par \textbf{Event:} Something that occurs in a certain place during a particular interval of time with the participation of invited people in case of an appropriate weather. A social gathering or activity at the indicated time and in the indicated place .
\par \textbf{User:} A person registered in the system. Users can create and manage events and be invited to other users events. 

\par \textbf{Alloy:} Alloy is a language for describing structures and a tool for exploring them. 

\begin{center}
	\begin{tabular}{| c | l |}
		\hline
		\textbf{Abbr} & \textbf{Definition} \\ \hline
		UML & Unified Modeling Language \\ \hline
		GB & Gigabyte \\ \hline
		RAM & Random Access Memory \\ \hline
		EJB & Enterprise Java Beans  \\ \hline
		JSF & Java Server Faces  \\ \hline
		JVM & Java Virtual Machine \\ \hline
	\end{tabular}
\end{center}

\subsection{References}
\quad \begin{itemize}
         \item Star UML Guide, retrieved from http://staruml.sourceforge.net/docs/user-guide(en)/toc.html
         \item IEEE. IEEE Std 830-1998 IEEE Recommended Practice for Software Requirements Specifications. IEEE Computer Society, 1998.
       \end{itemize}

\subsection{Overview}
\quad There are mainly four more parts of the document. The first part after the Introduction part is the Overall Description. This section describes the general factors that affect our product and its requirements. The Specific Requirements part contains all of the software requirements to a level of detail sufficient to enable designers to design a system to satisfy those requirements, and testers to test that the system satisfies hose requirements. The fourth section is the Data Model and Description part. At this section, the domain of the software is explain with the class diagrams. At the end the conclusion part and the appendix part, which contains the alloy code and corresponding diagrams, take place.