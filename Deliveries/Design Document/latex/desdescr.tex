\newpage

\begin{center}
\section{Design description information content}
\end{center}

\subsection{Introduction}
\qquad Software Design Document of MeteoCAl project identifies how this system will be implemented and designed. MeteoCal project has a modular object-oriented structure. Furthermore a qualified interface will be designed in a way that system will represent events of the simulation successfully.
\subsection{DD identification}
\qquad This DD contains these features:
\begin{itemize}
  \item Summary
  \item Glossary
  \item Change history
  \item Date of issue and status
  \item Scope
  \item Issuing organization
  \item Authorship (responsibility or copyright information)
  \item References
  \item Context
  \item One or more design languages for each design viewpoint used
  \item Body
\end{itemize}
\subsection{Design stakeholders and their concerns}
\qquad Design stakeholders of MeteoCal project is the developer team of the system and their advisors. Design concern of the stakeholders is implementing the project in a modular structure according to Design Document. End product has to contain design features that are described at Design Document. Modular approach is essential for the project since there will be multi types of clients; namely, web client and mobile clients.
\subsection{Design views}
\qquad In this document contextual, composition, interface, logical, interaction and state dynamics view will be explained in next sections. Detailed description and diagrams about these views will clarify them. Each view is given with its corresponding viewpoint.
\subsection{Design Viewpoints}
\qquad Software design description identifies context, composition, interface, logical, interaction and state dynamics viewpoints. Context viewpoint specifies the system boundaries and actors interacting with the system. Composition viewpoint identifies the system modules, components, frameworks and system repositories. Interface viewpoint explains the interaction between different software interfaces. Logical viewpoint includes the detailed description of data design and class diagrams. Interaction viewpoint gives the sequence of events in the system and the state dynamics viewpoint models the system as a state machine and shows the state transitions and conditions on these transitions.
\subsection{Design rationale}
\qquad In this project, design choices are made according to performance concerns and integrability of the system. System has to be designed in a way that future models and features can be added and current models can be changed and updated independently. Stakeholders may have and request further requirements, therefore system parts have to be modular. Developers of the system has to document development process and use comments in their code frequently, so that in the future other developers may understand code and the structure of the system.
\subsection{Design languages}
\qquad In this project, Unified Modeling Language (UML) is selected as a part of design viewpoint and it will be used for clarifying design viewpoints.  